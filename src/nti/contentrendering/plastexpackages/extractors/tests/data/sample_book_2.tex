\chapter{Chapter 1}
\label{chapter:1}

This chapter consists of two sections.

\section{Section 1.1}
\label{section:1-1}

Inevitably, when the discussion turns to technology and professional development, the debate of in-person versus online education comes up. Many associations express concern about what moving online will do to their in-person educational opportunities, and whether online or in-person is better. 

Our answer is always - why can't you do both? It is not an either/or discussion. As humans, we need community and great things can happen when we gather to learn face-to-face. However, in-person training is not always practical or cost effective. Online learning offers opportunities that are just not possible with in-person training. Here are three ways we have seen associations blend their education offerings to deliver the most value to their members.

\section{Section 1.2}
\label{section:1-2}

Some associations fear that offering online training will be detrimental to their in-person gatherings. However, evidence indicates that increased availability of high quality learning options usually leads to greater overall membership participation. 

\chapter{Chapter 2}
\label{chapter:2}

This chapter consist of 2 sections.

\section{Section 2.1}
\label{section:2-1}

If your education program is built for only one type of learning, you may be missing out on gaining new members and increasing non-dues revenue. Thus, increasing education options, especially online, is likely to lead to increased participation and revenue.

\subsection{Sub Section 2.1.1}
\label{subsection:2-2-1}

The same training you offer at conferences may not work best online. While recorded webinars are the simplest way to scale up online courses, they aren't always the most effective. There are better ways to build online learning experiences, if you design accordingly. 

\subsection{Sub Section 2.1.2}
\label{subsection:2-2-2}

For example, you can increase retention by breaking videos into multiple, shorter segments and interlacing those videos with readings, surveys and interactive experiences. Also, there are many ways to harness the power of community with online learning Fostering discussion groups and community resource sharing within your learning community is one of the best ways to engage learners. It is ideal if you can foster community within your learning management system.

\section{Section 2.2}
\label{section:2-2}
Conferences and online education offerings can benefit from each other. For instance, you can expand on successful classes at conferences by offering "going deeper" classes online. Conversely, if you find an online course is popular, expand on it by offering a related option at your conference. In-person and online trainings can build and feed off each other to provide your members with a more well-rounded education experiences. 


